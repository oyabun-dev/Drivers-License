ALGORITHME: BISSEXTILE
VARIABLES annee : ENTIER
            reponse : CHAINE DE CARACTERES
DEBUT
    ECRIRE "Entrez une année"
    LIRE annee // 1980
    SI annee MOD 4 = 0 ALORS
        SI annee MOD 100 <> 0 ALORS
            reponse <- "bissextile"
        SINON                                
            SI annee MOD 400 = 0 ALORS
                reponse <- "bissextile"
            SINON
                reponse <- "non bissextile"
            FIN SI
        FIN SI
    SINON
        reponse <- "non bissextile"
    FIN SI
    ECRIRE "L'année " annee " est " reponse"."
FIN


% //Une année est bissextile (contient 366 jours) si elle est multiple de 4, sauf les années de début de
% //siècle (qui se terminent par 00) qui ne sont bissextiles que si elles sont divisibles par 400. Exemples :
% //– 1980 et 1996 sont bissextiles car elles sont divisibles par 4
% //– 2000 est une année bissextile car elle est divisible par 400
% //– 2100 et 3000 ne sont pas bissextiles car elles ne sont pas divisibles par 400.




//Proposer un algorithme permettant de saisir des réels compris entre 0 et 20. Les réels qui ne sont pas entre 0 et 20 ne peunvent pas être saisis et si on veut arrêter la saisie, on tape -1. 
ALGORITHME: ENTRE0ET20
VARIABLES reel : REEL
            compteur : ENTIER
DEBUT
    compteur <- 0
    ECRIRE "Entrez un réel entre 0 et 20"
    LIRE reel
    TANT QUE reel <> -1 FAIRE
        SI reel >= 0 ET reel <= 20 ALORS
            compteur <- compteur + 1
        SINON
            ECRIRE "Erreur, le réel doit être compris entre 0 et 20"
        FIN SI
        ECRIRE "Entrez un réel entre 0 et 20"
        LIRE reel
    FIN TANT QUE
    ECRIRE "Vous avez saisi " compteur " réels entre 0 et 20"
FIN